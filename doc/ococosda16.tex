\documentclass[conference]{IEEEtran}

\usepackage{amsmath}
\usepackage{algorithm}
\usepackage{algpseudocode}
\usepackage{graphicx}

\makeatletter
\def\BState{\State\hskip-\ALG@thistlm}
\makeatother
   
\newcommand{\argmax}{\arg\!\max}

% correct bad hyphenation here
\hyphenation{op-tical net-works semi-conduc-tor}

\begin{document}

\title{Analyzing Differences in Indonesian Spontaneous and Dictated Speech}

\author {
    \IEEEauthorblockN {
        Cil Hardianto Satriawan\IEEEauthorrefmark{1},
        Dessi Puji Lestarti\IEEEauthorrefmark{2}
    }
    \IEEEauthorblockA {
        School of Electrical Engineering and Informatics\\
        Bandung Institute of Technology\\
        Bandung, Indonesia 40132\\
        \IEEEauthorrefmark{1}23515053@std.stei.itb.ac.id,
        \IEEEauthorrefmark{2}dessi@informatika.org
    }
}

\maketitle

\begin{abstract}
The accurate recognition of spontaneous speech is crucial in achieving practical speech recognition.
Due to the difficulties associated with the collection and labelling of spontaneous speech, a larger amount of dictated speech recorded from prepared transcripts is commonly utilized in the training of statistical-based speech recognition systems.
Unfortunately, such systems often suffer from poor spontaneous speech recognition performance.
In an effort to improve spontaneous recognition performance, we attempt to pinpoint the differences between spontaneous and dictated Indonesian speech.
At the phone level, we find that there are differences in the distribution and pronunciation of phones between conversational and written Indonesian.
More generally, conversational/spontaneous Indonesian is generally spoken faster and softer than its written/dictated form.
Using these differences as a starting point, we show that it is possible to classify Indonesian speech segments as being spontaneous or dictated based on a small number of simple features including segment duration and average log energy levels.
\end{abstract}

\begin{IEEEkeywords}
speech recognition, 
\end{IEEEkeywords}

\IEEEpeerreviewmaketitle

\section{Introduction}
The ability to accurately recognize spontaneous speech is a significant hurdle to the practical use of a speech recognition system, with many systems performing poorly.
This is seemingly in contrast to the recognition performance achievable on most state-of-the-art statistical-based systems on dictated speech tasks and limited domain spoken interactions.
In \cite{hoesen}, an Indonesian language speech recognition system was built that is capable of achieving upwards of 80\% accuracy for dictated speech but only around 65\% for spontaneous speech.
Other examples can be found in \cite{}, \cite{}, and \cite{}.

This poor performance may be partly attributed to the fact that a large proportion of the data used to train models is derived from dictated instead of spontaneous speech.
Recording and labelling large amounts of new spontaneous speech is a difficult and costly process, whereas existing Indonesian language resources are scarce or non-existant.
In addition, although spontaneous recognition is an important component of practical speech recognition, research has only relatively recently shifted in this direction.

Spontaneous and dictated speech also differ in various ways.
It has been proposed that spontaneous speech is less ordered both linguistically and acoustically \cite{nakamura1}, and differs spectraclly from dictated speech \cite{nakamura2}.
Spontaneous speech contains filled pauses, repairs, hesitations, repetitions, and disfluencies, which are largely absent in dictated speech.
The exact differences vary by language, with some languages exhibiting a strong tonal difference between spontaneous and dictated speech \cite{spanish}.
Speech disfluencies also differ between language and speaker; each language, dialect, or speaker may have their own set of specific disfluencies and patterns.

A number of methods have been developed to improve spontaneous speech recognition performance.
In the model adaptation approach, a large amount of dictation data is adapted to a small amount of spontaneous data while training the acoustic model, using a method such as Maximum A Priori (MAP).
In the second approach, acoustic and language models for spontaneous and dictated speech are trained separately, with a method to switch or weigh models on the fly.
Model switching relies on a reliable way to differentiate between spontaneous and dictated speech that is independent of the specific acoustic models.

In either case, it is instructive to understand how acoustic differences between spontaneous and dictated speech manifest and how to describe them.
At the temporal level, speech audio data is represented as audio samples taken at a particular sampling rate.
A constant number of (overlapping) audio samples are processed in a frame to obtain low level spectral, cepstral, and energy features.
A segment, comprised of a varying number of such frames, represents a single phoneme, the smallest linguistic unit.
Linguistically, phonemes are organized into words and sentences.
For the purposes of our acoustic analysis, however, linguistic units are ignored and we instead consider utterances as sequences of segments.
In addition to the temporal level, a number of high level 'metadata' features are also considered acoustically important.
These features are the specific speaker of an utterance, their gender, dialect, and age.

\section{Related Works}

Wang, et al. \cite{wang} proposed an incremental ensemble method called Accuracy-Weighted Ensembles (AWE).
It is a weighted majority ensemble algorithm with the weight is calculated based on error reduction analysis.
Each classifier $c_i$ is given a weight reversely proportional to the expected error MSE$_i$.
The weight is then calculated as $w_i = \text{MSE}_i - \text{MSE}_r$ where $\text{MSE}_r$ is the mean square error of a random classifier.

Another ensemble method for incremental learning is Streaming Ensemble Algorithm (SEA) \cite{street}.
It combines decision tree models using majority-voting.
It does not weight each classifier and vote accordingly.
It is based on investigation that weighting had little or no effect on ensemble accuracy.

Both ensemble methods are applied on concept drifting environment.
We concluded that concept drift has the same meaning as what we called OOV condition in the sense of the underlying concept, which we considered as terms or words, does not reflected in the predictive features \cite{street}.
The concept can also change without warning.
Zang, et al. \cite{zang} shows that AWE performs better than SEA.

For multilabel classification task, there are two main approach: problem transformation and algorithm adaptation \cite{tsoumakas}.
In problem transormation approach, some commonly methods are Binary Relevance (BR), Label Powerset (LP), and Calibrated Label Ranking (CLR).
Some examples of algoritm adaptation methods are Adaboost.MH, Adaboost.MR, and MLkNN.

In BR, multilabel dataset is transformed into $n$ single-label dataset where $n$ is the number of label \cite{prajapati}.
CLR applies the comparisons among the pair of labels exist in dataset, then builds both pair and single-label models.
LP method changes multilabel classification into a multiclass classification \cite{demb} by adding some new single-labels for each unique set of labels exists in multilabel dataset \cite{prajapati}.

The first two methods of algorithm adaptation are the extension of AdaBoost algorithm that search for accurate classification rules by combining multiple weak hypothesis \cite{dharmadhikari}.
MLkNN is the adaptation of kNN algorithm which uses kNN separately for each label then determines the k closest instances to the tested one.
Adaboost.MH and Adaboost.MR perform well in terms of one error, coverage, ranking loss and average precision, while MLkNN performs better in terms of hamming loss \cite{zhang}.

\section{Perisalah Corpus}

The Perisalah Corpus is a collection of spontaneous and dictated speech recordings obtained for the development of the Perisalah Speech Recognition System \cite{} in collaboration with PT. Inti, an Indonesian public telecommunications company.
The corpus was designed to sample a large variety of different age groups and local dialects for the purpose of building models suited to recognising high level government meetings at both the national and regional levels.
The read/dictation part of the corpus was build to be lexically balanced with respect to written Indonesian.

\subsection{Corpus Overview}

The corpus, after cleaning, consists of 297 native Indonesian speakers, with on average 275 dictated and 61 spontaneous utterances per speaker, totalling 81240 dictated and 18210 spontaneous utterances, respectively.
On average, dictated utterances are 6.54 seconds long and spontaneous utterances 8.33 seconds long for a total of 147.6 and 42.1 hours of dictated and spontaneous speech, respectively.

Demographically, speakers were taken from both genders roughly equally, with 143 male speakers and 154 female speakers.
The distribution of dialects was unfortunately limited by the availability of speakers at the time of recording, with a disproportionate number of Sundanese speakers.
The sampled dialects were Javanese, Sundanese, Minang, Batak, Betawi/Melayu, Balinese, Sulawesi, and Maluku/Papua.
Roughly 57\% of Indonesians live on Java, where the Javanese, Sundanese, and Betawi/Melayu dialects originate.

Two age groups were defined, "young" being 40 years of age of under, and "old" being above 40 years of age.
242 speakers fell into the first category and 52 the latter, with 3 undefined.

AWE proposed by Wang, et al. \cite{wang} is used as ensemble method for our approach.
Its weighting method is based on expected error of classifier.
We compute the mean square error of classifier $c_i$ by using formula
\[\text{MSE}_i = \frac{1}{|S|} \sum_{(x,y) \in S} (1 - f_i^y(x))^2\]
where $S$ is the latest chunk of training data, $(x,y)$ is a record in $S$ with $y$ is the true label, and $f_i^y(x)$ is the probability given by $c_i$ that $x$ is an instance of label $y$.
And then, the mean square error of each label according to its distribution is
\[\text{MSE}_r = \sum_{c} p(c)(1-p(c))^2\]
Finally, the weight $w_i$ of a classifier $c_i$ is computed with formula
\[w_i = \text{MSE}_r - \text{MSE}_i\]
$\text{MSE}_r$ does not contain useful information about the data.
It is only used as a threshold to discard classifiers from ensemble.
We eliminate classifiers that have negative weight, which means their error is equal or larger than $\text{MSE}_r$.

Adaboost.MH and Adaboost.MR is based on boosting ensemble method.
Even though they provide what we need: ensemble-multilabel classifier, we can not employ one of these methods because they do not handle incremental learning.
Since BR is the most widely used method \cite{santos}, has low computational complexity \cite{cherman}, simple, and easy to use, we use this as our multilabel classifier.
BR transforms multilabel classification problem by using $l$ single-label classifier, where $l$ is the number of label.
All single-label classifier is trained using the same training data but with different label.
When classifying documents, each single-label classifier classify the documents to label it handles.
The classification result is multilabel consisting of each single label.

\subsection{Corpus Analysis}

Suppose we have chunks of training data $S = {S_1, S_2, ..., S_n}$.
We perform indexing procedure that consists of case folding, tokenization, stop words elimination, stemming, and TF-IDF weighting.
Every chunks will be processed the same way by all of indexing procedure steps, except TF-IDF weighting.
It is because each chunk will have its own feature set.
That being said, we only need one case folder, tokenizer, stop words eliminator, and stemmer, but different TF-IDF transformer for each chunk.

In training step, each chunk of documents will be transformed into TF-IDF weights resulting in a TF-IDF transformer that consists of vocabulary or word set and their document frequency, depends on the contents of documents in the chunk.
After that, a classifier is trained using TF-IDF weights for each chunk.
We, then, weight each classifier in the ensemble using $\text{MSE}_r$ and $\text{MSE}_h$.
The resulting product is an ensemble consists of classifiers that accompanied by its weight and TF-IDF transformer.

There are two condition that makes us remove a classifier in the ensemble.
First, when there exists an ensemble member that has weight less or equal zero.
If a classifier has error larger than $\text{MSE}_r$, it means it does not perform well in classifying current training data.
We discard this classifier from ensemble because it is not good enough to be used in future classifying task, assuming the current training data represents what the next training data will be.
Second condition occurs if we have a limit to the number of ensemble members.
When we train a new classifier from new chunk of training data while the number of classifiers in the ensemble has already reached its limit, we remove a classifier from the ensemble and add the new classifier to it.
Classifier weight is used to determine which classifier we remove.
If there is $k$ limit of classifiers, we choose the top $k$ weighted classifiers, including the new one.
Because of the weight is reversly proportional to the classifier expected error, we can expect the top $k$ classifiers as the best classifiers.

In multilabel classification setup, the topology is not an ensemble of BR classifiers.
But instead, it is a BR classifier of ensemble where the ensemble members are single-label classifiers.
Fig. \ref{ensemble_of_br} and Fig. \ref{br_of_ensemble} illustrates ensemble of BR and BR of ensemble respectively.
If we have 10 labels and the ensemble has 4 members, there will be one BR classifier consists of 10 ensembles, each ensemble classifies one label.
And then, each ensemble will consist of 4 single-label classifier.
That means, the total number of classifiers we create is $10 \times 4 = 40$ classifiers.
We understand that the tradeoff of using this approach is there will be a lot classifiers.

\begin{figure}[!htb]
\centering
%\includegraphics[width=3.5in]{ensemble-of-br}
\caption{Ensemble of BR.}
\label{ensemble_of_br}
\end{figure}

\begin{figure}[!htb]
\centering
%\includegraphics[width=3.5in]{br-of-ensemble}
\caption{BR of ensemble.}
\label{br_of_ensemble}
\end{figure}

Note that what we remove from the ensemble is ensemble member.
If the member is BR, we will remove the BR classifier with all single-label classifiers it contains.
This is not right when the other single-label classifiers performs well.
We need a topology that allows us to remove only one single-label classifier without removing the others.
That is why we choose BR of ensemble topology where the member is single-label classifier.
It is allows us to remove single-label classifier from ensemble without removing the other.
With $L$ ensembles, we can contain them in a single BR to have a multilabel classifier of $L$ labels.

Each single-label classifier will be accompanied by one TF-IDF transformer.
But, in different ensemble, there may be more than one classifiers uses the same TF-IDF transformer.
For effiency, we should not use one-to-one relationship of TF-IDF transformer and classifier.
Instead, we use one-to-many relationship, which means one TF-IDF transformer can be used by many classifiers.
For the sake of simplicity, we do not consider this efficency aspect in the next explanation.
We treat accompanionship of TF-IDF transformer and classifier as one-to-one relationship.

Note that each TF-IDF transformer covers its own word set.
When OOV occurs, we can train a new TF-IDF transformer and a classifer using new training data that contains OOV words and add it to the ensemble.
That being said, we can always cover new words using this strategy, hence OOV condition can be handled.
Training procedure is summarized in Algorithm \ref{algo_train}.

\begin{algorithm}[!htb]
\caption{Training procedure}
\label{algo_train}
\begin{algorithmic}
\Procedure{Train}{}
    \BState \emph{input}:
        \State $S$: new chunk
        \State $k$: maximum number of classifiers
        \State $L$: labels
        \State $B$: Binary Relevance (BR) classifier of $|L|$ ensembles
    \BState \emph{algorithm}:
        \State fit TF-IDF transformer $f^{new}$ on $S$
        \State transform $S$ into $D^{new}$ by $f^{new}$
        \For {$l \in L$}
            \State train classifier $c_l^{new}$ on $D^{new}$ using label $l$
            \State compute weight $w_l^{new}$ of $c_l^{new}$
            \For {$c_l^i \in B_l$ (label $l$ ensemble in BR)}
                \State transform $S$ into $D_l^i$ by $f_l^i$
                \State apply $c_l^i$ on $D_l^i$ using label $l$ to derive $MSE_i$
                \State compute weight $w_l^i$ of $c_l^i$
            \EndFor
            \State $B_l \gets k$ of the top weighted classifiers in $B_l \cup \{c_l^{new}\}$
            \State remove negative weighted classifiers in $B_l$
        \EndFor
\EndProcedure
\end{algorithmic}
\end{algorithm}

\subsection{Segment Classification}

The first step is top-down procedure.
The document we want to classify is "sent" to the system, BR of ensembles.
The BR forwards this document to all ensembles.
Each ensembles forwards the document to every single-label classifier.
TF-IDF transformer in each classifier performs indexing to this document and then the classifier performs classification.

The process is continued with bottom-up procedure.
Each ensemble in the system votes the label class.
This is done by choosing a class that has maximum support, total weight of classifiers that predict class $y$.
The voting rule is formalized by
\[\hat{y} = \argmax_y \sum_{c \in C \rightarrow y} w_c\]
where $\hat{y}$ is the ensemble result of label class, $C \rightarrow y$ is classifiers that predict $y$, and $w_c$ is the weight of the classifier.
And then, the result of each ensemble is combined to make one multilabel class of the document.

This procedure is summarized in Algorithm \ref{algo_classify}

\begin{algorithm}[!htb]
\caption{Classification procedure}
\label{algo_classify}
\begin{algorithmic}
\Procedure{Classify}{}
    \BState \emph{input}:
        \State $s$: a document
        \State $L$: labels
        \State $B$: Binary Relevance (BR) classifier of $|L|$ ensembles
    \BState \emph{output}:
        \State $Y$: multilabel result
    \BState \emph{algorithm}:
        \For {$l \in L$}
            \State $W_0 \gets 0 , W_1 \gets 0$
            \For {$c_l^i \in B_l$ (label $l$ ensemble in BR)}
                \State transform $s$ into $d_l^i$ by $f_l^i$
                \State predict $y$ by $c_l^i$ on $d_l^i$
                \State $W_y \gets W_y + w_l^i$
            \EndFor
            \State $Y_l \gets \argmax_y W_y$
        \EndFor
        \State \Return $Y$
\EndProcedure
\end{algorithmic}
\end{algorithm}

\section{Experiment Results and Analysis}

Our data were taken from multilabel corpus constructed by Rahmawati \cite{rahma} and extended with two new datasets.
We consider one dataset as one chunk, that means there are 3 chunks.
We also have a test data consists of 100 documents that are different from the previous ones, also taken from \cite{rahma}.
All chunks and test data are multilabel data with 10 labels.
Data and label composition are shown in Table \ref{data_compos} and Table \ref{label_compos}.
The number of words are counted after stop words elimination and stemming performed.

\begin{table}[!htb]
\renewcommand{\arraystretch}{1.3}
\caption{Data composition}
\label{data_compos}
\centering
\begin{tabular}{|l|r|r|r|r|r|}
    \hline
     & Chunk A & Chunk B & Chunk C & \textbf{Total} & Testing\\
    \hline
    Documents & 690 & 1057 & 1479 & 3226 & 100\\
    \hline
    Cardinality & 1.448 & 1.018 & 1.014 & 1.108 & 1.3\\
    \hline
    Words & 5548 & 7788 & 8419 & 12286 & 1253\\
    \hline
\end{tabular}
\end{table}

\begin{table}[!htb]
\renewcommand{\arraystretch}{1.3}
\caption{Label composition}
\label{label_compos}
\centering
\begin{tabular}{|c|r|r|r|r|r|}
    \hline
    Label & Chunk A & Chunk B & Chunk C & \textbf{Total} & Testing\\
    \hline
    1 & 100 & 92 & 150 & 342 & 7\\
    \hline
    2 & 100 & 120 & 150 & 370 & 25\\
    \hline
    3 & 100 & 115 & 150 & 365 & 17\\
    \hline
    4 & 100 & 108 & 150 & 358 & 6\\
    \hline
    5 & 100 & 91 & 150 & 341 & 19\\
    \hline
    6 & 100 & 109 & 150 & 359 & 8\\
    \hline
    7 & 100 & 107 & 150 & 357 & 7\\
    \hline
    8 & 100 & 109 & 150 & 359 & 26\\
    \hline
    9 & 100 & 114 & 150 & 364 & 5\\
    \hline
    10 & 100 & 111 & 150 & 361 & 10\\
    \hline
    \textbf{Total} & 1000 & 1076 & 1500 & 3576 & 130\\
    \hline
\end{tabular}
\end{table}

\section{Conclusion and Future Works}

Our experiment compares the performance of our method against other methods.
There are four methods in this experiment: batch, incremental, ensemble, and ensemble-incremental.
Base classifier in every methods is Naive Bayes classifier.
We choose Naive Bayes because it is fast and simple, and also produce the second best performance beside SVM in Rahmawati experiment \cite{rahma}.
We use the same metric used by Rahmawati to evaluate each method performance, which is sample average f-measure.
The methods are evaluated using 10-fold cross validation and our test data.

In batch experiment, the first chunk is learned and the result is a BR classifier.
This is the first iteration.
After that, the data from first chunk is appended with the second chunk to make a new classifier, discarding the old classifier.
After the second iteration, we do the same process in third iteration with the third chunk.
In every iteration we perform, we evaluate its performance.

In other methods, the chunks are trained incrementally.
In incremental method, we train the BR of Naive Bayes classifier incrementally chunk by chunk.
Note that Naive Bayes classifier can be trained incrementally by only updating the distribution $P(y)$ and $P(x_i|y)$.
The next two methods use our method.
In the first one, ensemble method, we only perform our method as is.
In ensemble-incremental method, we furtherly employ incremental learning ability of Naive Bayes to existing classifiers in the ensemble.

We also perform chunk sorting in order to fit AWE weighting scheme analysis.
Classifier weighting in AWE is estimated from its expected prediction error on the test examples, assuming class distribution of the most recent training data is closest to class distribution of test data \cite{wang}.
To meet this assumption, we sort our chunks by evaluating three classifiers that are trained independently using each chunk.

We masure processing time and memory usage of each method to see whether incremental and ensemble methods use less resource.
Implementation of ensemble methods that is used for this experiment is the efficient one, which has one-to-many relationship between TF-IDF transformer and classifier.
OOV condition is also analyzed by calculating the number of words that are covered and not covered by the classifiers, its effect on class prediction, and a sample of document that has OOV words.
Experiment results are explained in next chapter.

Natural language processing (NLP) tools we use for tokenization and stemming tasks is INA-NLP by Purwarianti \cite{ayu}.
We use stop words list taken from Tala \cite{tala}.
For machine learning task, we use Python programming language and Scikit-learn \cite{sklearn}.

\section{Result}

The result of 10-fold cross validation is shown in Table \ref{result_k}.
Note that the result of first iteration in all methods are identical because they are actually the same classifier, trained with the same data.
No re-learn procedure has happened in incremental, ensemble, and ensemble-incremental classifier.
Incremental-trained classifier will be similar to batch-trained classifier.
Also, ensemble and ensemble-incremental will consist of one member which is a classifier that similar to batch-trained one.

\begin{table}[!htb]
\renewcommand{\arraystretch}{1.3}
\caption{Cross validation result}
\label{result_k}
\centering
\begin{tabular}{|l|c|c|c|}
\hline
 & Iteration 1 & Iteration 2 & Iteration 3\\
\hline
Batch & 0.785 & 0.768 & 0.787\\
\hline
Incremental & 0.785 & 0.735 & 0.778\\
\hline
Ensemble & 0.785 & 0.775 & 0.768\\
\hline
Ensemble-incremental & 0.785 & 0.775 & 0.776\\
\hline
\end{tabular}
\end{table}

Ensemble and ensemble-incremental method has the same F1 score in first and second iteration.
This happens because there are only two members in the ensemble, so the voting will be to choose one from two classifiers.
This means, a classifier that has larger weight is the same one between ensemble and ensemble-incremental method.

The result against testing data is shown in Table \ref{result_test}.
Here, like in cross validation, all methods has the same performance is first iteration.
Ensemble-incremental is always outperform ensemble-only method in validation and testing.
This concludes that existing classifiers in ensemble are also need to be re-trained with new training data in order to make them adapt with the most up-to-date condition.

\begin{table}[!htb]
\renewcommand{\arraystretch}{1.3}
\caption{Testing result}
\label{result_test}
\centering
\begin{tabular}{|l|c|c|c|}
\hline
 & Iteration 1 & Iteration 2 & Iteration 3\\
\hline
Batch & 0.778 & 0.798 & 0.823\\
\hline
Incremental & 0.778 & 0.796 & 0.8\\
\hline
Ensemble & 0.778 & 0.71 & 0.756\\
\hline
Ensemble-incremental & 0.778 & 0.71 & 0.776\\
\hline
\end{tabular}
\end{table}

To see the effect of meeting AWE weighting scheme assumption, we use f-measure to estimate closeness of chunk class distribution to our test data.
A chunk which has larger F1 score is assumed to indicate its class distribution is closer to test data.
Classifier F1 score when trained with Chunk A is 0.778, 0.71 with Chunk B, and 0.728 with Chunk C.
Therefore, we use Chunk B in first iteration, Chunk C in the second, and Chunk A in the last iteration.

Classifiers performance using test data after the chunks has been sorted is shown in Table \ref{result_test_sorted}.
We can see that both ensemble methods perform better than before sorted.
Ensemble-incremental method even outperform batch method in the last iteration.
This shows that the assumption of the most recent chunk as the closest to class distribution of test data affects performance.

\begin{table}[!htb]
\renewcommand{\arraystretch}{1.3}
\caption{Sorted chunk testing result}
\label{result_test_sorted}
\centering
\begin{tabular}{|l|c|c|c|}
\hline
 & Iteration 1 & Iteration 2 & Iteration 3\\
\hline
Batch & 0.71 & 0.76 & 0.823\\
\hline
Incremental & 0.71 & 0.745 & 0.811\\
\hline
Ensemble & 0.71 & 0.728 & 0.791\\
\hline
Ensemble-incremental & 0.71 & 0.728 & 0.834\\
\hline
\end{tabular}
\end{table}

Each methods computational time and memory usage is shown in Table \ref{profile}.
We can see that even though both ensembles take more time, their processing time is proportional to how much documents is used in the training.
This also applied in incremental method, since they only use current chunk and do not use any previous chunk.
Note that the last chunk used in third iteration contains the least number of documents.
Incremental, ensemble, and ensemble-incremental methods took less time than previous iteration.
Ensemble methods trained much longer because there are more classifiers and more process, such as weighting and TF-IDF transforming.
In contrast, batch method processing time is proportional to total number of documents its used, chunk size did not affect it.
In memory usage we can also see batch method uses more memory in the last iteration than other methods.
Incremental and ensemble methods memory usage of training data depends on chunk size.
Ensemble methods require more memory than incremental method because they contain more classifiers.

\begin{table}[!htb]
\renewcommand{\arraystretch}{1.3}
\caption{Training time and memory usage}
\label{profile}
\centering
\begin{tabular}{|l|c|c|c|}
\hline
 & Iteration 1 & Iteration 2 & Iteration 3\\
\hline
\textbf{Training Time} & & & \\
\hline
Batch & 0.42 & 0.881 & 1.09\\
\hline
Incremental & 0.42 & 0.47 & 0.291\\
\hline
Ensemble & 0.702 & 1.523 & 1.241\\
\hline
Ensemble-incremental & 0.702 & 1.59 & 1.356\\
\hline
\textbf{Memory Usage} & & & \\
\hline
Batch & 20762 & 41953 & 52132\\
\hline
Incremental & 20590 & 23224 & 14331\\
\hline
Ensemble & 20604 & 29432 & 24568\\
\hline
Ensemble-incremental & 20604 & 29683 & 25090\\
\hline
\end{tabular}
\end{table}

To investigate OOV occurence, we analyzed words contained in each chunk.
The number of words that appear in chunk A and also appear in chunk B is 3882, and that makes the total of unique words of their union is 9454.
It means, 1666 words are OOV words.
That union and the chunk C has 5587 same words.
It has 3867 OOV words from that union compared to words consisting in the chunk C.

OOV occurrence can be seen in prediction of a sample document in our test data with title "Partai Buruh Dinilai Sangat Mungkin Terbentuk" [Labor Party is Considered Very Likely to Formed].
The word "kecimpung" [involved in] appears in that document, but it is not covered by incremental method classifier.
This OOV condition resulted in incorrect classification result.
Ensemble method correctly classify the document to label "social and cultural", but incremental method does not.
When we create a dummy document consisting only the word "kecimpung", we got the same result, ensemble method classifies it correctly but incremental method classifier does not.
This is an example of OOV can affect prediction performance.

\section{Conclusion and Further Works}

An ensemble approach to incrementally learn out of vocabulary words has been implemented in this experiment for multilabel classification of Indonesian news article.
Our approach is a combination of Accuracy-Weighted Ensemble (AWE) for ensemble classifier and Binary Relevance (BR) for multilabel classifier.
We provided explanation of the usage of both methods in order to employ them as one classifier system.

The results show that our ensemble methods could perform well when the last training data represent class distribution of test data the most.
Our approach is incremental method, in the sense that it does not use any previous data, making its processing time and memory consumption does not proportionally increased by total number of all documents.
OOV is also shown to have been handled using this approach.

Our method will be improved by making its processing time shorter.
Parallel implementation can be employed to address this problem.
There are some BR and AWE procedures that can be performed in parallel, since classifiers contained in them process data independently.
Feature selection can also be performed.
There are evidence that shows feature selection increases classifier performance for the same task \cite{rahma}.
Although this method still leaves some opportunity for improvements, we believe that this paper has covered the big picture of what our idea is.

\begin{thebibliography}{1}

\bibitem{dellwo}
V.~Delwo, A.~Leemann, M.J.~Kolly,
    "The recognition of read and spontaneous speech in local vernacular: The case of Zurich German,"
    Journal of Phonetics vol. 48, 2015, pp. 13-28.

\bibitem{silverman}
K.~Silverman, E.~Blaauw, J.~Spitz, J.~Pitrelli,
    "Towards Using Prosody in Speech Recognition/Understanding Systems: Differences Between Read and Spontaneous Speech",

\bibitem{liu}
G.~Liu, Y.~Lei, J.H.L.~Hansen,
    "Dialect Identification: Impact of Differences Between Read and Spontaneous Speech",
    18th European Signal Processing Conference (EUISPCO-2010), 2010.

\bibitem{furui1}
S.~Furui,
    "Spontaneous Speech Recognition and Summarization",

\bibitem{benzeguiba}
M.~Benzeguiba, R.~DeMori, O.~Deroo, S.~Dupon, T.~Erbes, D.~Jouvet, L.~Fissore, P.~Laface, A.~Mertins, C.~Ris, R.~Rose, V.~Tyagi, C.~Wellekens,
	"Automatic Speech Recognition and Speech Variability: a Review",
	Speech Communication 49:763-786, 2007.

\bibitem{asami}
T.~Asami, R.~Masamura, H.~Masataki, S.~Sakauchi,
    "Read and spontaneous speech classification based on variance of GMM supervectors",
    
\bibitem{nakamura2}
M.~Nakamura, I.~Koji, S.~Furui,
    "Differences between acoustic characteristics of spontaneous and read speech and their effects on speech recognition performance",
    Computer Speech and Language 22(2): 171-184, 2008.

\bibitem{sebastiani}
F.~Sebastiani,
    "Machine learning in automated text categorization,"
    ACM computing surveys (CSUR) 34.1 (2002): 1-47.

\bibitem{ayu}
A.~Purwarianti,
    "A non deterministic Indonesian stemmer,"
    Electrical Engineering and Informatics (ICEEI),
    2011 International Conference on. IEEE, 2011.

\bibitem{rahma}
D.~Rahmawati and M.L.~Khodra,
    "Automatic multilabel classification for Indonesian news articles,"
    Advanced Informatics: Concepts, Theory and Applications (ICAICTA),
    2015 2nd International Conference on. IEEE, 2015.

\bibitem{calandra}
R.~Calandra, et al.,
    "Learning deep belief networks from non-stationary streams,"
    Artificial Neural Networks and Machine Learning–ICANN 2012,
    Springer Berlin Heidelberg, 2012, 379-386.

\bibitem{muhlbaier}
M.D.~Muhlbaier and R.~Polikar,
    "An ensemble approach for incremental learning in nonstationary environments,"
    Multiple classifier systems, Springer Berlin Heidelberg, 2007, 490-500.

\bibitem{wang}
H.~Wang, F.~Wei, P.~Yu, and J.~Han,
    "Mining concept-drifting data streams using ensemble classifiers,"
    In Proceedings of the ninth ACM SIGKDD international conference on Knowledge discovery and data mining,
    pp. 226-235, ACM, 2003.
    
\bibitem{street}
W.~Street and Y.Kim,
    "A streaming ensemble algorithm (SEA) for large-scale classification," 
    Proceedings of the seventh ACM SIGKDD international conference on Knowledge discovery and data mining,
    ACM, 2001.

\bibitem{zang}
W.~Zang, et al.,
    "Comparative study between incremental and ensemble learning on data streams: case study."
    Journal Of Big Data 1.1 (2014): 1-16.

\bibitem{tsoumakas}
G.~Tsoumakas, I.~Katakis, and I.~Vlahavas,
    "Mining multi-label data,"
    in Data mining and knowledge discovery handbook,
    Springer, 2010, pp. 667–685.

\bibitem{prajapati}
P.~Prajapati, A.~Thakkar, and A.~Ganatra,
    "A Survey and Current Research Challenges in Multi-Label Classification Methods,"
    Int. J. Soft Comput., vol. 2, 2012.

\bibitem{demb}
K.~Dembczynski, W.~Waegeman, W.~Cheng, and E.~Hüllermeier,
    "On label dependence in multi-label classification,"
    in Workshop proceedings of learning from multi-label data,
    2010, pp. 5–12.

\bibitem{dharmadhikari}
S.C.~Dharmadhikari, M.~Ingle, and P.~Kulkarni,
    "A comparative analysis of supervised multi-label text classification methods,"
    2011.

\bibitem{zhang}
M.L.~Zhang and Z.H.~Zhou,
    "ML-KNN: A lazy learning approach to multi-label learning,"
    Pattern Recognit., vol. 40, no. 7, pp. 2038–2048, 2007.

\bibitem{santos}
A.~Santos, A.~Canuto, and A.~Neto,
    "A comparative analysis of classification methods to multi-label tasks in different application domains,"
    Int. J. Comput. Inform. Syst. Indust. Manag. Appl, vol. 3,
    pp. 218–227, 2011.

\bibitem{cherman}
E.A.~Cherman, M.C.~Monard, and J.~Metz,
    "Multi-label problem transformation methods: a case study,"
    CLEI Electron. J., vol. 14, no. 1, p. 4, 2011.
    
\bibitem{tala}
F.Z.~Tala,
    "A study of stemming effects on information retrieval in Bahasa Indonesia,"
    Institute for Logic, Language and Computation Universeit Van Amsterdam (2003).
    
\bibitem{sklearn}
F.~Pedregosa et al.,
    "Scikit-learn: Machine learning in Python,"
    The Journal of Machine Learning Research 12 (2011): 2825-2830.

\end{thebibliography}



% CONTEKAN:


% An example of a floating figure using the graphicx package.
% Note that \label must occur AFTER (or within) \caption.
% For figures, \caption should occur after the \includegraphics.
% Note that IEEEtran v1.7 and later has special internal code that
% is designed to preserve the operation of \label within \caption
% even when the captionsoff option is in effect. However, because
% of issues like this, it may be the safest practice to put all your
% \label just after \caption rather than within \caption{}.
%
% Reminder: the "draftcls" or "draftclsnofoot", not "draft", class
% option should be used if it is desired that the figures are to be
% displayed while in draft mode.
%
%\begin{figure}[!t]
%\centering
%\includegraphics[width=2.5in]{myfigure}
% where an .eps filename suffix will be assumed under latex, 
% and a .pdf suffix will be assumed for pdflatex; or what has been declared
% via \DeclareGraphicsExtensions.
%\caption{Simulation results for the network.}
%\label{fig_sim}
%\end{figure}

% Note that the IEEE typically puts floats only at the top, even when this
% results in a large percentage of a column being occupied by floats.


% An example of a double column floating figure using two subfigures.
% (The subfig.sty package must be loaded for this to work.)
% The subfigure \label commands are set within each subfloat command,
% and the \label for the overall figure must come after \caption.
% \hfil is used as a separator to get equal spacing.
% Watch out that the combined width of all the subfigures on a 
% line do not exceed the text width or a line break will occur.
%
%\begin{figure*}[!t]
%\centering
%\subfloat[Case I]{\includegraphics[width=2.5in]{box}%
%\label{fig_first_case}}
%\hfil
%\subfloat[Case II]{\includegraphics[width=2.5in]{box}%
%\label{fig_second_case}}
%\caption{Simulation results for the network.}
%\label{fig_sim}
%\end{figure*}
%
% Note that often IEEE papers with subfigures do not employ subfigure
% captions (using the optional argument to \subfloat[]), but instead will
% reference/describe all of them (a), (b), etc., within the main caption.
% Be aware that for subfig.sty to generate the (a), (b), etc., subfigure
% labels, the optional argument to \subfloat must be present. If a
% subcaption is not desired, just leave its contents blank,
% e.g., \subfloat[].


% An example of a floating table. Note that, for IEEE style tables, the
% \caption command should come BEFORE the table and, given that table
% captions serve much like titles, are usually capitalized except for words
% such as a, an, and, as, at, but, by, for, in, nor, of, on, or, the, to
% and up, which are usually not capitalized unless they are the first or
% last word of the caption. Table text will default to \footnotesize as
% the IEEE normally uses this smaller font for tables.
% The \label must come after \caption as always.
%
%\begin{table}[!t]
%% increase table row spacing, adjust to taste
%\renewcommand{\arraystretch}{1.3}
% if using array.sty, it might be a good idea to tweak the value of
% \extrarowheight as needed to properly center the text within the cells
%\caption{An Example of a Table}
%\label{table_example}
%\centering
%% Some packages, such as MDW tools, offer better commands for making tables
%% than the plain LaTeX2e tabular which is used here.
%\begin{tabular}{|c||c|}
%\hline
%One & Two\\
%\hline
%Three & Four\\
%\hline
%\end{tabular}
%\end{table}


% Note that the IEEE does not put floats in the very first column
% - or typically anywhere on the first page for that matter. Also,
% in-text middle ("here") positioning is typically not used, but it
% is allowed and encouraged for Computer Society conferences (but
% not Computer Society journals). Most IEEE journals/conferences use
% top floats exclusively. 
% Note that, LaTeX2e, unlike IEEE journals/conferences, places
% footnotes above bottom floats. This can be corrected via the
% \fnbelowfloat command of the stfloats package.



% trigger a \newpage just before the given reference
% number - used to balance the columns on the last page
% adjust value as needed - may need to be readjusted if
% the document is modified later
%\IEEEtriggeratref{8}
% The "triggered" command can be changed if desired:
%\IEEEtriggercmd{\enlargethispage{-5in}}



% that's all folks
\end{document}
